\documentclass{article}

% ================== Packages ==================
\usepackage{tikz}
\usepackage{quantikz}
\usepackage{braket}
\usepackage[margin=1in]{geometry}
\usepackage{enumitem}
\usepackage{float}
\usepackage{amsmath}

\allowdisplaybreaks
\tikzset{slice/.append style={draw=black, text=black}}

% ================== Title ==================
\title{Introduction to Quantum Information and Quantum Computing\\Assignment 2}
\author{Manuel Santos -- 2019231352}
\date{January 2026}

\begin{document}

\maketitle

% ================== Introduction ==================
\section*{Introduction}
This report presents the solution to Assignment 2 of the Introduction to Quantum Information and Quantum Computing course.

The goal of this assignment is to implement Grover's Algorithm.
In Grover’s algorithm, an oracle marks a secret computational basis state by inverting the sign of its amplitude, while leaving all other computational basis states unchanged.
The goal of the algorithm is to identify the secret state using the smallest possible number of oracle queries.

% ================== 1 ==================
\section{Preparation of Grover's Oracle}

In order to implement Grover's algorithm, we first need to prepare the oracle state.
For this assignment, we will prepare the oracle state $\ket{101}$ as the one to be marked by the oracle.
The circuit for preparing the oracle state is shown in Figure~\ref{fig:oracle_prep}.

\begin{figure}[H]
    \centering
    \begin{quantikz}
        \lstick{$q_0:$} & \qw      & \ctrl{1} & \qw      & \qw \\
        \lstick{$q_1:$} & \gate{X} & \ctrl{1} & \gate{X} & \qw \\
        \lstick{$q_2:$} & \gate{H} & \targ{}  & \gate{H} & \qw
    \end{quantikz}
    \caption{Circuit for Grover's Oracle Preparation}
    \label{fig:oracle_prep}
\end{figure}

In order to understand how this circuit works, we can represent it in a equivalent way, as shown in Figure~\ref{fig:oracle_deconstruction}.

\begin{figure}[H]
    \centering
    \begin{quantikz}
        \lstick{$q_0:$} & \qw      & \slice[color=black]{} & \ctrl{1} & \qw      & \qw &\\
        \lstick{$q_1:$} & \gate{X} & & \ctrl{1} & \gate{X} & \qw &\\
        \lstick{$q_2:$} & \qw      & & \gate{H} & \targ{}  & \gate{H} & \qw
    \end{quantikz}
    \caption{Circuit for Grover's Oracle Preparation}
    \label{fig:oracle_deconstruction}
\end{figure}

% \begin{figure}[H]
%     \centering
%     \begin{quantikz}
%         \lstick{$q_0:$} & \gate{H} & \slice[style=black]{$\ket{\mathrm{Barrier}}$} & & \ctrl{1} & & \ctrl{2} & & \slice[style=black]{$\ket{\mathrm{GHZ}}$} & & \\
%         \lstick{$q_1:$} &          &                                              & & \targ{}  & &           & & & & \\
%         \lstick{$q_2:$} &          &                                              & &          & & \targ{}  & & & &
%     \end{quantikz}
%     \caption{Circuit for Grover's Oracle Preparation}
%     \label{fig:oracle_prep}
% \end{figure}

\begin{enumerate}[label=(\alph*)]
    \item
    \begin{enumerate}[label=\roman*)]
        \item
        Let $\ket{\mathrm{Barrier}}$ denote the state at the first barrier. Since the Hadamard gate is applied only to qubit $A$ (the least significant qubit in the $\ket{\mathrm{CBA}}$ ordering), we obtain
        \[
        \ket{000}
        \xrightarrow{\mathrm{H}_A}
        \ket{00}\!\left(\frac{\ket{0}+\ket{1}}{\sqrt{2}}\right)
        =
        \frac{1}{\sqrt{2}}\ket{000}
        +
        \frac{1}{\sqrt{2}}\ket{001}
        = \ket{\mathrm{Barrier}}.
        \]

        \item
        Starting from $\ket{\mathrm{Barrier}}$, we apply a CNOT gate with $A$ as control and $B$ as target, followed by a CNOT with $A$ as control and $C$ as target:
        \[
        \ket{\mathrm{Barrier}}
        \xrightarrow{\mathrm{CNOT}_{A,B}}
        \frac{1}{\sqrt{2}}\ket{000}
        +
        \frac{1}{\sqrt{2}}\ket{011}
        \xrightarrow{\mathrm{CNOT}_{A,C}}
        \frac{1}{\sqrt{2}}\ket{000}
        +
        \frac{1}{\sqrt{2}}\ket{111}
        = \ket{\mathrm{GHZ}}.
        \]
    \end{enumerate}

    At the first barrier, the system is in a superposition of $\ket{000}$ and $\ket{001}$. After the CNOT operations, this superposition is extended across all three qubits, producing the maximally entangled GHZ state.

    \item
    Measuring Alice’s qubit collapses the state:
    \begin{itemize}
        \item If Alice measures $0$, Bob and Charlie collapse to $\ket{00}$
        \item If Alice measures $1$, Bob and Charlie collapse to $\ket{11}$
    \end{itemize}
    From Bob’s or Charlie’s local perspective, the outcomes remain random until classical communication occurs.

    \item
    Since $\ket{\mathrm{GHZ}}$ is an equal superposition of $\ket{000}$ and $\ket{111}$, we expect each outcome with probability $1/2$. This behavior is confirmed by simulation, as shown in Figure~\ref{fig:ex1_c}.

    \begin{figure}[H]
        \centering
        \includegraphics[width=0.5\textwidth]{Images/1_c.png}
        \caption{Simulation results for 1024 shots}
        \label{fig:ex1_c}
    \end{figure}
\end{enumerate}

% ================== 2 ==================
\section{Independent Operations}

After receiving their qubits, Alice, Bob, and Charlie independently apply either
\[
M_X = H \quad \text{or} \quad M_Y = HS^\dagger,
\]
followed by measurement.

\begin{enumerate}[label=(\alph*)]
    \item
    When all three apply $M_X$, the resulting state $\ket{\mathrm{XXX}}$ is produced by the circuit in Figure~\ref{fig:mx_circuit}.

    \begin{figure}[H]
        \centering
        \begin{quantikz}[wire types={q,q,q,c}]
            \lstick{$A:$} & \gate{H} & \ctrl{1} & \ctrl{2} & \slice{$\ket{\mathrm{GHZ}}$} & & \gate{H} & & \meter{} & \\
            \lstick{$B:$} &          & \targ{}  &          &                              & & \gate{H} & &          & \meter{} \\
            \lstick{$C:$} &          &          & \targ{}  &                              & & \gate{H} & &          &          \meter{}
        \end{quantikz}
        \caption{Circuit XXX: all parties apply $M_X = H$}
        \label{fig:mx_circuit}
    \end{figure}

    Applying the Hadamard gates yields
    \[
    \ket{\mathrm{XXX}}
    =
    \frac{1}{2}
    \big(
    \ket{000}
    +
    \ket{011}
    +
    \ket{101}
    +
    \ket{110}
    \big).
    \]

    Simulation results are shown in Figure~\ref{fig:ex2_XXX}.

    \begin{figure}[H]
        \centering
        \includegraphics[width=0.5\textwidth]{Images/2_a.png}
        \caption{Simulation results for circuit XXX (1024 shots)}
        \label{fig:ex2_XXX}
    \end{figure}

    \item
    The possible outcomes are $\ket{000}$, $\ket{011}$, $\ket{101}$, and $\ket{110}$. In all cases, the sum $a+b+c$ is even.

    \item
    When exactly one party applies $M_X$ and the others apply $M_Y$, only four outcomes occur, and the sum $a+b+c$ is always odd.

    \item
    The outcomes for each configuration are:
    \begin{itemize}
        \item YYX, YXY, XYY:
        \[
        (C,B,A) = (0,0,1), (0,1,0), (1,0,0), (1,1,1).
        \]
    \end{itemize}
    These results confirm the GHZ correlations.
\end{enumerate}

\end{document}
